\chapter{总结与展望}
\label{chap:future}

\section{本文工作的总结}
核磁共振成像是现代医学最常用的成像手段,可以非侵入性地提供人体内组织的图像。提高核磁共振的成像速度一直是研究人员关心的问题,而利用压缩感知理论加速MR成像的方法已经被广泛地应用于临床和研究。在压缩感知的模型中,为动态MR图像选择合适的稀疏项是一个重要的问题,直接影响了重建图像的质量。核磁共振指纹是定量磁共振成像的新方法,可以在单次数据采集中同时获得多种组织参数。但是由于MRF的字典生成速度和模板匹配的速度一般十分耗时,对计算机的内存和存储都是巨大的负担。因此如何快速并精确地生成字典并进行字典匹配是一个亟待解决的问题。本文针对以上问题,研究的内容和创新点主要体现在以下几个方面:

1. 针对动态MR图像,利用压缩感知和图像分解的思想,提出了基于二阶时空TGV和核范数的重建模型。模型将图像分解为低秩部分和稀疏部分,其中低秩部分用核范数约束,稀疏部分用二阶时空TGV约束。核范数用来建模动态图像中时间方向高度相关的背景部分,并且可以很好地去除空间伪影;而TGV泛函用来表示图像中的光滑部分,可以在保证重建图像边界清晰的同时减少重建图像中的阶梯效应。文章利用Primal-Dual算法来求解模型,并给出了保证算法收敛的范数估计。为了减少算法的计算时间,文章也利用了图形处理单元来加速程序。文章针对体模、心脏灌注与胸部DCE-MRI图像,对比了四种最前沿的针对动态MR重建模型在不同采样模式和不同采样率下的表现。实验结果表明,相对于其他四个模型,我们提出的模型在不同采样模式和不同采样率下,可以更好地消除空间伪影并且保证图像边缘的清晰,对于胸部DCE-MRI图像的效果尤其显著。

2. 如何选择胸部DCE-MRI压缩感知模型中时间方向的稀疏项一直是未知的,为此我们针对胸部DCE-MRI图像,比较了5种不同的时间方向的稀疏项,并对重建结果进行了定量分析。这5种稀疏项分别为Fourier变换、Haar小波变换、全变差、二阶TGV和核范数。所有模型都使用FISTA算法进行求解。实验结果表明,核范数可以得到最高的信噪比,而TV/二阶TGV可以得到最精确地定量分析。因此,对于胸部DCE-MRI, 选择TV/二阶TGV作为稀疏项可以更好地重建病灶部分,而选择核范数则可以整体提高图像的信噪比。

3. 针对MRF中字典生成和模板匹配速度慢的缺点,我们利用图形处理单元进行MRF字典的生成和匹配,并开发了一款开源软件snapMRF。软件基于NVIDIA公司的底层框架CUDA,使用C语言编写。snapMRF可以快速并准确地重建参数图,并且适用于不同的MR序列。相比于其他MRF开源软件,如EPG-X(基于CPU的MATLAB语言编写)和pnp-mrf(基于CPU的C语言编写),snapMRF的字典生成速度提高了10--1000倍, 字典匹配速度提高了10--100倍。对于较小的字典,可以达到实时的效果。

\section{下一步一些展望}
1. 本论文重点研究了压缩感知在动态MR图像重建中的应用,没有考虑其他类型的医学图像,如计算机断层扫描(CT)、超声(US)、正电子成像(PET)等。我们可以将本文的模型应用到上述图像中,这也是值得考虑的事情。

2. 第四章中所使用到的程序均用MATLAB编写,并且在CPU上运行。因此程序运行速度较慢,通常重建一例图像需要10--20分钟。可以考虑使用GPU来进行加速,由于我们在第三章和第五章已经有了使用GPU编程的经验,这一点应该是很容易实现的。

3. 近几年,深度学习在图像处理领域中变得十分火热,并且以后研究将其应用在压缩感知和MRF的参数图重建中。所以如何利用深度神经网络来进行压缩感知的重建和MRF参数图的重建也是一个十分有趣的问题。
