\begin{publications}{99}
%\item[] {\songti\zihao{4}\bf{攻读博士学位期间已发表的论文:}}
%\item \textbf{第一作者},Magnetic Resonance in Medicine(SCI小类核医学2区,大类医学2区,TOP期刊)。
%\item \textbf{第一作者},Magnetic Resonance Imaging(SCI小类核医学3区,大类医学3区)。
%\item \textbf{第一作者},International Journal of Biomedical Imaging(EI,已检索)。
%
%\vspace{1.0cm}
%\item[] {\songti\zihao{4}\bf{攻读博士学位期间参加的科学研究情况:}}
%\setcounter{enumiv}{0}
%\item 201$\cdot$年主持南京理工大学优秀博士研究培养基金。
%\item 201$\cdot$年参与国家自然科学基金重大研究计划培养项目一项。


\item[] {\songti\zihao{4}\bf{攻读博士学位期间已发表的论文:}}
\item Dong Wang, David S. Smith and Xiaoping Yang. Dynamic MR Image Reconstruction using TGV and Low-rank Decomposition. Magnetic Resonance in Medicine (SCI 二区). 2019. doi: 10.1002/mrm.28064.
\item Dong Wang, Jason Ostenson, and David S. Smith. snapMRF: GPU-Accelerated Magnetic Resonance Fingerprinting Dictionary Generation and Matching using Extended Phase Graphs. Magnetic Resonance Imaging (SCI 三区). 2019. doi: 10.1016/j.mri.2019.11.015.
\item Dong Wang, Lori R. Arlinghaus, Thomas E. Yankeelov, Xiaoping Yang, and David S. Smith. Quantitative Evaluation of Temporal Regularizers in Compressed Sensing Dynamic Contrast Enhanced MRI of the Breast. International Journal of Biomedical Imaging (EI). 2017. doi: 10.1155/2017/7835749.

\vspace{1.0cm}
\item[] {\songti\zihao{4}\bf{攻读博士学位期间参加的科学研究情况:}}
\setcounter{enumiv}{0}
\item \texttt{}主持了南京理工大优秀博士研究培养基金(2013-2015)
\item \texttt{}参与了国家自然科学基金重大研究计划培育项目“高阶非线性偏微分方程图像模型及其基础算法”(项目编号:91330101, 2014.01-2016.12,杨孝平教授主持)

\end{publications}
