\begin{abstract}
核磁共振成像(magnetic resonance imaging,MRI)是医学临床和研究中最常用的成像方式,可以非侵入式地获取人体内组织的信息,用于指导病灶检测与临床诊断。但由于MR成像受到物理和生理上的限制,其成像速度一般很慢,尤其是动态MR图像(dynamic MRI,dMRI)。因此,如何在保证图像质量的情况下,提高成像速度一直是研究人员最关心的问题。

压缩感知(compressed sensing)是近十年来图像处理领域的研究热点之一。作为一种新兴的采样理论,压缩感知可以突破传统Nyquist采样定理的瓶颈,用远少于Nyquist采样所需的数据精确地重建出图像。压缩感知理论有三个主要的部分,即图像稀疏性、随机下采样和非线性重建算法。具体来说,压缩感知是在假设图像在某个变换域稀疏的前提下,对图像进行随机下采样,利用非线性算法重建出图像。因此,压缩感知可以加快成像速度,减少存储压力。同时压缩感知可以加快成像速度,提高图像的时间分辨率和空间分辨率。自从压缩感知理论诞生以来,就被广泛应用于医学成像领域,尤其是核磁共振成像中。对于MR图像的压缩感知模型而言,最重要一点是寻找适合于MR图像的稀疏表示。此外,对于动态MR图像而言,时间分辨率往往和空间分辨率同样重要,因此如何选择合适的时间方向的稀疏表示,提高动态MR图像的时空分辨率一直是一个具有挑战的问题。

定量磁共振成像技术(quantitative MRI,qMRI)是利用某些特殊的成像序列,测量组织参数的成像技术,如横向弛豫时间($T_1$)、纵向弛豫时间($T_2$)、质子密度等。相比于传统的定性(qualitative)MR图像,定量MRI从客观定量的角度研究人体组织,起到了帮助诊断与评估治疗的作用。其中磁共振动态对比增强 (dynamic contrast enhanced MRI, DCE-MRI) 是近些年来定量成像的研究热点,其通过获取注入对比剂前后的图像,经过一系列的计算分析,得到定量或半定量的参数,用于指导临床诊断。目前,压缩感知理论已经被应用在加速DCE-MRI成像中,但对于胸部DCE-MRI图像而言,如何选择和评估时间方向的稀疏项一直是未知的。

磁共振指纹(magnetic resonance fingerprinting, MRF)是定量MRI的新方法,可以在单次数据采集中同时获得多种组织参数。磁共振指纹主要分为三个部分,即预定义字典生成、信号采集和模式识别。具体来说,给定某个MR序列,首先用已有的MR成像的数学模型模拟生成一个包含不同参数的组织随时间演化的字典,然后选择某个模式识别算法将采集得到的信号(也被称为指纹)与字典中的原子进行匹配,从而重建出组织的参数图像。由于信号的长度一般在1,000以上,字典中的原子个数也经常达到10,000以上,字典生成与匹配的所消耗的时间通常达到几十分钟甚至几小时。因此,如何快速并且精确地生成字典并且进行字典匹配是一个亟待解决的问题。

针对以上问题,本文研究的工作以及创新点主要有以下三个方面:

1. 针对动态MR图像,利用压缩感知和图像分解的思想,提出了基于二阶时空TGV(total generalized variation)和核范数的重建模型。模型将图像分解为低秩部分和稀疏部分,其中低秩部分用核范数约束,稀疏部分用二阶时空TGV约束。核范数用来建模动态图像中时间方向高度相关的背景部分,可以很好地去除空间伪影;而TGV泛函用来表示图像中的光滑部分,可以在保证重建图像边界清晰的同时减少重建图像中的阶梯效应。文章利用Primal-Dual算法来求解模型,并给出了保证算法收敛的范数估计。为了减少算法的计算时间,我们利用了图形处理单元(graphics processing unit, GPU)来加速程序。我们针对体模、心脏灌注与胸部DCE-MRI图像,对比了四种最前沿的针对动态MR重建模型在不同采样模式和不同采样率下的表现。数值实验结果表明,相比于其他四个模型,我们提出的模型在不同采样模式和不同采样率下,可以更好地消除空间伪影并且保证图像边缘的清晰,对于胸部DCE-MRI图像的效果尤其显著。

2. 针对胸部DCE-MRI图像,比较了5种不同的时间方向的稀疏项在压缩感知重建模型中的表现,并对重建结果进行了定量分析。这5种稀疏项分别为Fourier变换(Fourier transform)、Haar小波变换(Haar wavelet transform)、TV(total variation)、二阶TGV和核范数(nuclear norm)。所有模型均使用FISTA(fast iterative shrinkage threshold algorithm)快速算法进行求解。数值实验结果表明,基于核范数的模型可以得到最高的信噪比,而基于TV/二阶TGV的模型可以得到最精确地定量分析。因此,对于胸部DCE-MRI, 选择TV/二阶TGV作为稀疏项可以更好地重建病灶部分,而选择核范数则可以提高图像的整体信噪比。

3. 针对MRF重建参数图的过程中字典生成与匹配速度慢的问题,利用图形处理单元进行MRF字典的生成与匹配,并开发了一款开源程序snapMRF。程序基于NVIDIA公司的底层框架CUDA(compute unified device architecture),使用C语言编写。snapMRF可以快速并准确地重建参数图,并且适用于不同的MR序列。相比于其他MRF开源程序,如EPG-X(基于CPU的MATLAB语言编写)和PnP-MRF(基于CPU的C语言编写),snapMRF的字典生成速度提高了10--1000倍, 字典匹配速度提高了10--100倍。对于较小的字典,可以达到实时的效果。另外,程序也给出了6种经典MR序列的单元测试,保证对于不同输入的MR序列,字典生成与匹配的正确性。

\keywords{核磁共振成像,动态核磁共振,压缩感知,变分模型,稀疏表示,磁共振指纹成像,图像分解,核范数,TGV,定量分析,图形处理单元}
\end{abstract}


\begin{englishabstract}
Magnetic resonance imaging (MRI) is the most commonly used imaging methodology in medical clinical research, which can non-invasively acquire in vivo tissue information and help to guide lesion diagnosis and detection. However, as the speed of MRI is fundamentally limited by physical and physiological constraints, the imaging speed of MRI is rather slow, especially for dynamic MRI (dMRI). Thus, how to accelerate the imaging speed without degenerating the image quality has always been the most concerned issue for MR researchers.

Compressed sensing has been a hot topic for the last decade in image processing. As the state-of-the-art sampling theory, compressed sensing breaks through the bottleneck of traditional Nyquist sampling theorem and is able to accurately reconstruct the image using only a small number of sampling values -- much smaller than the Nyquist sampling theorem. Compressed sensing has three main components, which are sparse representation, random undersampling and nonlinear reconstruction algorithm, respectively. Specifically, given that the image is sparse in a certain transform domain, compressed sensing demonstrates that it is possible to reconstruct the image through a small number of random linear measures using a nonlinear reconstruction algorithm. Therefore compressed sensing has the potential to accelerate imaging speed as well as decrease storage burden, and has been widely used in medical imaging, especially in MRI. Compressed sensing can accelerate the whole procedure of MRI and improve both spatial and temporal resolution. For compressed sensing based MR reconstruction models, the most essential procedure is to find a proper sparsifying transform. Furthermore, for dynamic MRI, temporal resolution is often as important as spatial resolution. Therefore, how to select a proper temporal sparse representation for dynamic MR images has always been a challenging problem.

Quantitative MRI (qMRI) is the technology of measuring tissue parameters, such as $T_1$, $T_2$ and proton density, using certain MR sequences. In contrast to traditional qualitative MRI, qMRI studies human tissues from an objective and quantitative perspective, which can help to guide diagnosis and assess treatment. DCE-MRI (dynamic contrast enhanced MRI) has been a hot topic in qMRI these years, which can guide the clinical diagnosis and provide semiquantitative or quantitative parameters by acquiring and analyzing the images before and after the injection of contrast agent. Nowadays compressed sensing theory has been adopted to accelerate DCE-MRI reconstruction, but for breast DCE-MRI, how to select and assess temporal sparsifying transforms is unknown.

Magnetic resonance fingerprinting (MRF) is the state-of-the-art quantitative imaging technique, which is able to achieve multiple tissue parameters in one single acquisition. MRF consists of three main parts, which are the generation of pre-defined dictionary, signal acquisition and pattern recognition, respectively. More specifically, given a certain MR sequence, a dictionary of simulated signal time courses is first generated for anticipated combinations of tissue properties using an appropriate signal model, and then pattern recognition is used to match each voxel in the acquisition (also called fingerprinting) to the dictionary to reconstruct the final parameter maps. As the signal length is often longer than 1,000, and the number of atoms in the dictionary is often over 10,000, it usually takes minutes or even hours to generate dictionaries and perform matching. Therefore, how to fast and accurately generate dictionaries and perform matching is an urgent problem.

Based on the above problems, the main work and innovation of the thesis can be listed as follows:

1. For dynamic MR images, we propose a novel decomposition based model exploiting second-order spatio-temporal total generalized variation (TGV) and the nuclear norm for compressed sensing dynamic MR reconstruction. The nuclear norm can model time-coherent background and perform well in removing spatial artifacts, and the TGV functional represents the smooth part well and can preserve the edges as well as reduce staircase effects. We first employ the Primal-Dual algorithm to solve the proposed model and then give the norm estimation for the convergence condition. We also implement the proposed model using GPU (graphics processing unit) in CUDA C to accelerate the MATLAB code. Numerical experiments on PINCAT, cardiac perfusion and breast DCE-MRI datasets indicate that the proposed model outperforms the state-of-the-art methods in both suppressing the spatial artifacts and preserving the edges under different acceleration factors and different sampling schemes, especially for breast DCE-MRI datasets.

2. For DCE-MRI of the breast, we compare five different sparse regularizers in the temporal direction and perform quantitative analysis. The five temporal regularizers are the Fourier transform, the Haar wavelet transform, total variation (TV), second-order TGV and the nuclear norm, respectively. All the models are solved using FISTA (fast iterative shrinkage thresholding algorithm). Numerical experiments demonstrate that the nuclear norm provides the highest SER while TV/TGV brings the most accurate parameter analysis. Therefore we recommend using TV/TGV as the temporal constraint in compressed sensing reconstructions of breast DCE-MRI.

3. Due to the long running time in MRF, we propose to use GPU to accelerate both dictionary generation and matching, and release an open source online program called snapMRF. snapMRF is based on the framework CUDA (compute unified device architecture) of NVIDIA and written in C language. snapMRF can fast and accurately reconstruct parameter maps and apply to different MR sequences. Compared with other online MRF programs, such as EPG-X (based on MATLAB, CPU) and PnP-MRF (based on C, CPU), snapMRF accelerates dictionary generation by $10-1000\times$ and matching by $10-100\times$. For dictionaries with small sizes, snapMRF can perform in a real-time manner. Besides, snapMRF provides the unit tests of 6 classic MR sequences, ensuring the correctness of dictionary generation and matching. 


\englishkeywords{MRI, Dynamic MRI, Compressed Sensing, Variational Models, Sparse Representation, MRF, Image Decomposition, Nuclear Norm, TGV, Quantitative Analysis, GPU}
\end{englishabstract}